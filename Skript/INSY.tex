
\documentclass[12pt]{amsart}
\usepackage{geometry} % see geometry.pdf on how to lay out the page. There's lots.
\geometry{a4paper} % or letter or a5paper or ... etc
% \geometry{landscape} % rotated page geometry

% See the ``Article customise'' template for come common customisations

\title{Informationssysteme}
\author{Peter Kovar}
\date{} % delete this line to display the current date
\usepackage{listings}


%%% BEGIN DOCUMENT
\begin{document}

\maketitle
\tableofcontents

\section{Basic Script}
\lstinputlisting[language=SQL]{pizza_3ahitm.sql}

\section{\\Joins}
Nachdem das Basic-Skript in die Datenbank geladen wurde k\"onnen wir mit den Joins beginnen.
\subsection{Wer hatte welche Pizza}
\texttt{\\ \\select kunden.vorname, kunden.nachname, pizzas.name}
\texttt{\\from bestellungen}
\texttt{\\join kunden on bestellungen.kunden\_id  = kunden.id}
\texttt{\\join pizzas on bestellungen.pizzas\_id  = pizzas.id;}

\subsection{Wer hatte welche Pizza mit Preis}
\texttt{\\ \\select kunden.vorname, kunden.nachname, pizzas.name, pizzas.preis}
\texttt{\\from bestellungen}
\texttt{\\join kunden on bestellungen.kunden\_id  = kunden.id}
\texttt{\\join pizzas on bestellungen.pizzas\_id  = pizzas.id;}

\subsection{Wieviel ist der Umsatz pro Kunde}
\texttt{\\ \\select kunden.vorname, kunden.nachname, sum(pizzas.preis)}
\texttt{\\from bestellungen}
\texttt{\\join kunden on bestellungen.kunden\_id  = kunden.id}
\texttt{\\join pizzas on bestellungen.pizzas\_id  = pizzas.id}
\texttt{\\group by kunden.id;}

\end{document}